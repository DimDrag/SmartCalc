% Options for packages loaded elsewhere
\PassOptionsToPackage{unicode}{hyperref}
\PassOptionsToPackage{hyphens}{url}
%
\documentclass[
]{article}
\usepackage{amsmath,amssymb}
\usepackage{iftex}
\ifPDFTeX
  \usepackage[T1]{fontenc}
  \usepackage[utf8]{inputenc}
  \usepackage{textcomp} % provide euro and other symbols
\else % if luatex or xetex
  \usepackage{unicode-math} % this also loads fontspec
  \defaultfontfeatures{Scale=MatchLowercase}
  \defaultfontfeatures[\rmfamily]{Ligatures=TeX,Scale=1}
\fi
\usepackage{lmodern}
\ifPDFTeX\else
  % xetex/luatex font selection
\fi
% Use upquote if available, for straight quotes in verbatim environments
\IfFileExists{upquote.sty}{\usepackage{upquote}}{}
\IfFileExists{microtype.sty}{% use microtype if available
  \usepackage[]{microtype}
  \UseMicrotypeSet[protrusion]{basicmath} % disable protrusion for tt fonts
}{}
\makeatletter
\@ifundefined{KOMAClassName}{% if non-KOMA class
  \IfFileExists{parskip.sty}{%
    \usepackage{parskip}
  }{% else
    \setlength{\parindent}{0pt}
    \setlength{\parskip}{6pt plus 2pt minus 1pt}}
}{% if KOMA class
  \KOMAoptions{parskip=half}}
\makeatother
\usepackage{xcolor}
\usepackage{longtable,booktabs,array}
\usepackage{calc} % for calculating minipage widths
% Correct order of tables after \paragraph or \subparagraph
\usepackage{etoolbox}
\makeatletter
\patchcmd\longtable{\par}{\if@noskipsec\mbox{}\fi\par}{}{}
\makeatother
% Allow footnotes in longtable head/foot
\IfFileExists{footnotehyper.sty}{\usepackage{footnotehyper}}{\usepackage{footnote}}
\makesavenoteenv{longtable}
\setlength{\emergencystretch}{3em} % prevent overfull lines
\providecommand{\tightlist}{%
  \setlength{\itemsep}{0pt}\setlength{\parskip}{0pt}}
\setcounter{secnumdepth}{-\maxdimen} % remove section numbering
\ifLuaTeX
  \usepackage{selnolig}  % disable illegal ligatures
\fi
\IfFileExists{bookmark.sty}{\usepackage{bookmark}}{\usepackage{hyperref}}
\IfFileExists{xurl.sty}{\usepackage{xurl}}{} % add URL line breaks if available
\urlstyle{same}
\hypersetup{
  hidelinks,
  pdfcreator={LaTeX via pandoc}}

\author{}
\date{}

\usepackage[russian]{babel}
\begin{document}

\subsection{Документация проекта SmartCalcv
1.0}\label{ux434ux43eux43aux443ux43cux435ux43dux442ux430ux446ux438ux44f-ux43fux440ux43eux435ux43aux442ux430-smartcalcv-1.0}

\begin{itemize}
\tightlist
\item
  Приложение поддерживает три режима - \textbf{стандартный (smart
  calc)}, \textbf{кредитный калькулятор (loan calc)} и
  \textbf{депозитный калькулятор (deposit calc)}. Переключаться между
  режимами можно с помощью клавиш F1, F2 и F3 соответственно.
\end{itemize}

\subparagraph{Режим Smart
calc:}\label{ux440ux435ux436ux438ux43c-smart-calc}

\begin{itemize}
\item
  Для ввода операндов и операторов можно пользоваться как мышью, так и
  клавиатурой. Чтобы открыть подменю тригонометрических функций, можно
  воспользоваться сочетанием клавиш Gtrl+T, остальных функций - Gtrl+F.
  Чтобы закрыть упомянутые подменю, нужно нажать клавишу Esc.
\item
  Если при нажатии кнопки равно (=), в выражении допущена синтаксическая
  ошибка (например, в выражении есть непарная скобка), в поле выражения
  будет отображено ``SYNTAX ERROR''.
\item
  Если при нажатии кнопки равно (=), в выражении допущена математическая
  ошибка (например, в выражении присутствует деление на ноль), в поле
  выражения будет отображено ``MATH ERROR''.
\item
  В данном режиме есть возможность ввода переменной. После окончания
  ввода математического выражения, следует нажать кнопку равно. Весь
  лишний функционал пропадёт и пользователю будет предложено ввести
  значение переменной. После ввода нужно так же нажать кнопку равно.
\item
  В данном режиме есть возможность просмотра графика введённого
  выражения. Для этого нужно включить режим графика и начать ввод
  выражения. График будет обновляться автоматически по мере ввода
  выражения. Если введённое выражение некорректно - график будет
  отображать функцию y=0.
\item
  Подробнее все сочетания клавиш описаны в списке ниже.
\item
  Список сочетаний клавиш:

  \begin{longtable}[]{@{}
    >{\raggedright\arraybackslash}p{(\columnwidth - 2\tabcolsep) * \real{0.7895}}
    >{\centering\arraybackslash}p{(\columnwidth - 2\tabcolsep) * \real{0.2105}}@{}}
  \toprule\noalign{}
  \begin{minipage}[b]{\linewidth}\raggedright
  Действие
  \end{minipage} & \begin{minipage}[b]{\linewidth}\centering
  Сочетание клавиш
  \end{minipage} \\
  \midrule\noalign{}
  \endhead
  \bottomrule\noalign{}
  \endlastfoot
  Квадратный корень & S+Q+R+T \\
  Возведение в степень & \^{} \\
  Очистка поля выражения & Delete \\
  Удаление одного оператора/операнда & Backspace \\
  Открывающая скобка & ( \\
  Закрывающая скобка & ) \\
  Остаток от деления & M+O+D \\
  Деление & / \\
  Умножение & * \\
  Сложение (в т.ч. унарное) & + \\
  Вычитание (в т.ч. унарное) & - \\
  Переменная X & x \\
  Точка & . \\
  Вычисление выражения & = \\
  Включить режим графика & Ctrl+G \\
  Открыть подменю тригонометрических функций & Ctrl+T \\
  Открыть подменю других функций & Ctrl+F \\
  Синус & S+I+N \\
  Косинус & C+O+S \\
  Тангенс & T+A+N \\
  Арксинус & A+S+I+N \\
  Арккосинус & A+C+O+S \\
  Арктангенс & A+T+A+N \\
  Десятичный логарифм & L+O+G \\
  Натуральный логарифм & L+N \\
  Переключение на режим \textbf{``Стандартный калькулятор'' (smart
  calc)} & F1 \\
  Переключение на режим \textbf{``Кредитный калькулятор'' (loan calc)} &
  F2 \\
  Переключение на режим \textbf{``Депозитный калькулятор'' (deposit
  calc)} & F3 \\
  \end{longtable}
\end{itemize}

\subparagraph{Режим Loan
calc:}\label{ux440ux435ux436ux438ux43c-loan-calc}

\begin{itemize}
\item
  Чтобы вычислить платежи, пользователю нужно заполнить следующие поля:

  \begin{itemize}
  \tightlist
  \item
    Loan type - вид платежей (аннуитетные/дифференцированные)
  \item
    Loan amount - начальное тело кредита
  \item
    Loan term - срок, на который берётся кредит
  \item
    Interest rate - процентная ставка
  \end{itemize}
\item
  После заполнения упомянутых полей, нужно нажать кнопку Calculate. В
  следующих полях будет отображен результат

  \begin{itemize}
  \tightlist
  \item
    Monthly payment - ежемесячный платёж (при дифференцированном -
    наибольший и наименьшие платежи)
  \item
    Total payment - общая выплата
  \item
    Total interest - переплата по кредиту
  \item
    В таблице ниже будет представлен список платежей, в котором указаны
    номер и сумма платежа, а также остаток долга после платежа.
  \end{itemize}
\item
  Список сочетаний клавиш:

  \begin{longtable}[]{@{}lc@{}}
  \toprule\noalign{}
  Действие & Сочетание клавиш \\
  \midrule\noalign{}
  \endhead
  \bottomrule\noalign{}
  \endlastfoot
  Вычислить & = \\
  Очистить все поля & Delete \\
  \end{longtable}
\end{itemize}

\subparagraph{Режим Deposit
calc:}\label{ux440ux435ux436ux438ux43c-deposit-calc}

\begin{itemize}
\item
  Чтобы вычислить выплаты, пользователю нужно заполнить следующие поля:

  \begin{itemize}
  \tightlist
  \item
    Initial - изначальная сумма вклада
  \item
    Length - срок размещения
  \item
    Interest rate - процентная ставка
  \item
    Tax rate - налоговая ставка (если нет - выставить значение 0)
  \item
    Compounding - периодичность выплат
  \item
    Capitalization - капитализация процентов
  \item
    Таблица ниже отвечает за пополнения и частичные снятия. Их
    заполнение опционально.
  \end{itemize}
\item
  После заполнения упомянутых полей, нужно нажать кнопку Calculate. В
  следующих полях будет отображен результат.

  \begin{itemize}
  \tightlist
  \item
    Total interest - начисленные проценты
  \item
    Total tax - сумма налога
  \item
    Ending balance - сумма на вкладе к концу срока
  \end{itemize}
\item
  Список сочетаний клавиш:

  \begin{longtable}[]{@{}lc@{}}
  \toprule\noalign{}
  Действие & Сочетание клавиш \\
  \midrule\noalign{}
  \endhead
  \bottomrule\noalign{}
  \endlastfoot
  Вычислить & = \\
  Очистить все поля & Delete \\
  \end{longtable}
\end{itemize}

\end{document}
